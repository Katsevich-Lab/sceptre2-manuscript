\documentclass[12pt]{article}
\usepackage{amsfonts}
\usepackage{amsmath}
\usepackage{graphicx} 
\usepackage{float}
\usepackage[caption = false]{subfig}
\usepackage{/Users/timbarry/Documents/optionFiles/mymacros}

\begin{document}

\bibliographystyle{unsrt}
\bibliography{/Users/timbarry/Documents/optionFiles/library.bib}

\noindent
Tim

\begin{center}
\textbf{A strategy for generating negative control pairs}
\end{center}

I describe a strategy for generating undercover pairs for use in the undercover analysis. The method takes the following inputs:

\begin{itemize}
\item The number of negative control gRNAs $N_\textrm{grna}$. Label the NT gRNAs $1, 2, \dots, N_\textrm{grna}$.
\item The $N_\textrm{cell} \times N_\textrm{gene}$ matrix of gene expressions and the $N_\textrm{cell}$-dimensional gRNA-to-cell assignment vector.
\item The number of pairs to generate $N_\textrm{pairs}$.
\item The undercover group size $k \leq N_\textrm{grna}/2$.
\item The minimum number of treatment cells $N_\textrm{trt}$ and control cells $N_\textrm{cntrl}$ used for pairwise QC.
\end{itemize}

The method constructs a set of undercover groups; each group is mapped to the same number of genes (give or take one gene). All gene-gRNA pairs have at least $N_\textrm{trt}$ treatment cells and $N_\textrm{cntrl}$ control cells. We proceed as follows.
\\ \\
\textbf{Initialization step: Tabulate the number of of cells with nonzero expression for each individual NT gRNA and gene.} First, we compute an $N_\textrm{grna} \times N_\textrm{gene}$ matrix $M$, where entry $(i,j)$ gives the number of cells containing NT gRNA $i$ with nonzero expression of gene $j$. We can easily construct this matrix either in memory or out-of-core. We then proceed in a sequence of rounds to construct the pairs.
\\ \\
\textbf{Round 1.} 

\underline{Step a}. We construct $r_1 := \lfloor N_\textrm{NT}/k\rfloor$ NT gRNA groups $G^1_1, \dots, G^1_{r_1}$ such that each NT gRNA group $G^1_i$ contains $k$ unique NT gRNAs. Algorithmically, we construct $G^1_1, \dots, G_{r_1}^1$ by sampling $k$ elements from $x = \{1, \dots, N_\textrm{grna}\}$ without replacement $\lfloor N_\textrm{NT}/k \rfloor$ times. To prepare for a subsequent step, we also initialize an empty set $D$ and add $G^1_1, \dots, G^1_{r_1}$ to the set $D$. 

\underline{Step b}. Next, for a given NT gRNA group, we determine the set of genes (which we call ``valid genes'') to which that NT gRNA group could be paired so that the resulting pairs pass the pairwise QC threshold. Let $v^1_i$ be the number of valid genes for gRNA group $G^1_i$. We can calculate $v^1_i$ by iterating over the matrix $M$.

\underline{Step c}. We determine whether there are enough valid genes (across undercover gRNA groups) such that that the total number of undercover gRNA group-gene pairs exceeds the threshold $N_\textrm{pairs}.$ Define $v_\textrm{min} = \min_{i \in \{1, \dots, r_1\}} v_i.$ We check if
\begin{equation}
v_\textrm{min} \geq \lfloor N_\textrm{pairs}/r_1 \rfloor + 1.
\end{equation}
If this equation holds, then we proceed to step d; otherwise, we proceed to round 2.

\underline{Step d}. Define $b = \lfloor N_\textrm{pairs}/r_1 \rfloor$, and define $l = N_\textrm{pairs} - r_1 \lfloor N / r_1 \rfloor.$ Note that $l < r_1$. Define $a_1 = \dots = a_l  = b + 1$, and define $a_{l+1} \dots a_{r_1} = b$. Observe that $$\sum_{i=1}^{r_1} a_i = l(b+1) + (r_1 - l)b= r_1 b + l = N_\textrm{pairs}.$$ Furthermore, observe that $a_i \leq b + 1 \leq v_\textrm{min}.$ Let $a_i$ be the number of genes that we sample for undercover gRNA group $G^1_i$. We sample these $a_i$ genes without replacement from the ``valid'' genes for $G^1_i$. 
\\ \\
\textbf{Round 2}.
\underline{Step a}. We construct $r_2$ NT gRNA groups $G^2_1, \dots, G^2_{r_2}$ such that each NT gRNA group $G^2_i$ contains $k$ unique NT gRNAs. Importantly, $G^2_1, \dots, G^2_{r_2}$ are constructed such that $G^2_i \notin D$, i.e., the $G^2_i$s are distinct from the $G^1_i$s. We construct the $G^2_i$s in the following way.

First, we sample $k$ elements without replacement from $x = \{1, \dots, N_\textrm{grna}\}$ to form $G^2_1$. We check if $G^2_1$ is present in $D$; if so, we start the process again. If not, we proceed, sampling $k$ elements from the elements that remain in $x$. We stop sampling when either fewer than $k$ elements remain or we have sampled $\binom{N_\textrm{grna}}{k}$ total undercover gRNA groups.

\underline{Step b}. For gRNA groups $G^2_1, \dots, G^2_{r_2}$, we compute the number of valid genes $v^2_1, \dots, v^2_{r_2}$ to which each gRNA group can be paired.

\underline{Step c}. We determine if there are enough valid genes across undercover gRNA groups OR we have exhausted the total number of undercover gRNA groups. We check the equation
$$ v_\textrm{min} \geq \lfloor N_\textrm{pairs}/(r_1 + r_2) \rfloor + 1,$$ where $v_\textrm{min}$ is taken over $G^1_1, \dots, G_{r_1}^1, G^2_1, \dots, G^2_{r_2}$. If this inequality is satisfied (or if we have exhausted all possible undercover groups), we proceed to round 3. Otherwise, we proceed to step d.

\underline{Step d}. Define $b = \lfloor N_\textrm{pairs}/(r_1 + r_2) \rfloor$, and define $l = N_\textrm{pairs} - (r_1 + r_2) \lfloor N / (r_1 + r_2) \rfloor$. Define $a_1 \dots, a_l = b+1$ and $a_{l+1} = \dots = a_{r_1 + r_2} = b$. Sample this many genes per undercover gRNA group. 


\end{document}
