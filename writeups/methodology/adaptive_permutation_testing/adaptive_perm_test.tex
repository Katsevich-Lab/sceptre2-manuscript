\documentclass[12pt]{article}
\usepackage{amsfonts}
\usepackage{amsmath}
\usepackage{graphicx} 
\usepackage{algorithm2e}
\usepackage{float}
\usepackage[caption = false]{subfig}
\usepackage{/Users/timbarry/Documents/optionFiles/mymacros}
\RestyleAlgo{ruled} 
\SetKwInOut{Input}{input}\SetKwInOut{Output}{output}
\newcommand{\bs}[1]{\boldsymbol{#1}}

\begin{document}

\section{A two-step adaptive permutation procedure}

We describe an adaptive, two-step permutation testing procedure that we call the ``checkpoint permutation test.'' The checkpoint permutation test is not new. However, we formalize the procedure, demonstrate its correctness, and precisely characterize its power-computation trade-off, which enables the principled selection of method hyperparameters in practice.

\begin{algorithm}
	\SetKwData{rankone}{r$_1$} \SetKwData{ranktwo}{r$_2$}	\SetKwData{p}{p}
	\caption{Left-tailed checkpoint permutation test}\label{alg:checkpoint_perm_one_sided}
	\Input{Number of null test statistics $B_1$ and $B_2$ to draw in the first and second rounds, respectively; threshold $c$.}
	\Output{Normalized rank \p of the ground truth statistic among the null statistics.}
	\BlankLine
	Compute ground truth test statistic $t$ and null statistics $y_1, \dots, y_{B_1}$\;
	\rankone $\leftarrow$ $\textrm{rank}(t; \{y_1, \dots, y_{B_1}\})$\;
	\eIf{$\rankone  \leq c$}{
		Compute fresh null statistics $z_1, \dots, z_{B_2}$\;
		\ranktwo $\leftarrow \textrm{rank}(t; \{z_1, \dots, z_{B_1}\})$\;
		\p $\leftarrow \ranktwo/(B_2 + 1)$
	}{
		\p $\leftarrow \rankone/(B_1 + 1)$
	}
\end{algorithm}

We begin with the left-tailed checkpoint permutation test (Algorithm \ref{alg:checkpoint_perm_one_sided}). Let $t \in \R$ be the ground truth test statistic. For $B_1 \in \N$, let $\mathcal{Y} = \{y_1, \dots, y_{B_1}\}$ be the set of null test statistics drawn during the first round. We compute the rank $r_1$ of $t$ among the $y_i$s:
$$ r_1 := \textrm{rank}(t; \mathcal{Y}) := | \{ y \in \mathcal{Y} : y \leq t \} | + 1,$$
i.e., $r_1$ is the number of elements in the set $\mathcal{Y} \cup \{ t \}$ that are less than or equal to $t$ (including $t$ itself). Note that $r_1$ takes values on $[B_1 + 1]$. Let $c \in [B_1 + 1]$ be a threshold. If $r_1 > c,$ then we set the normalized rank $p$ to $p = r_1/(B_1 + 1)$ and stop. Otherwise, we proceed to round two. For $B_2 \in \N$, let $\mathcal{Z} = \{ z_1, \dots, z_{B_2}\}$ be the set of null test statistics drawn during the second round. Let $r_2 := \textrm{rank}(t; \mathcal{Z})$ be the rank of $t$ among the $z_i$s. We set the normalized rank $p$ to $p = r_2/(B_2 + 1)$ and stop, returning $p$ to the user. Assume that there are no ties among elements within the set $\{ t \} \cup \mathcal{Y} \cup \mathcal{Z}$. (Otherwise, break ties by jittering.) Furthermore, assume that under the null hypothesis, the random variables in $\{ t \} \cup \mathcal{Y} \cup \mathcal{Z}$ are exchangeable.

\textbf{Proposition}. The support $\mathcal{S}$ of $p$ is $[B_1 + 1]/(B_1 + 1) \cup [B_2 + 1]/(B_2 + 1),$ where $0 < i \leq 1$ for all $i \in \mathcal{S}$. The distribution of $p$ is

\begin{multline*}
\P(P \leq x) = 
\end{multline*}

\textbf{Lemma}. We map the above onto a combinatorial problem. Suppose that we have $y_\textrm{tot}$ yellow balls, $g_\textrm{tot}$ green balls, and one black ball that we can arrange in any single-file order. Assume that each of the $(y_\textrm{tot} + g_\textrm{tot} + 1)!$ possible arrangements of the balls is equally probable. Let $Y_l \in \{0, \dots, y_\textrm{tot}\}$ (resp., $G_l \in \{0, \dots, g_\textrm{tot}\} $) denote the number of yellow (resp., green) balls that falls to the left of the black ball. Our goal is to compute the probability that a given number of green balls falls to the left of the black ball given that a certain number of yellow balls falls to the left of the black ball.

First, we compute the total number of configurations such that $Y_l = y_l$ and $G_l = g_l$, i.e., such that $y_l$ yellow balls and $g_l$ green balls fall to the left of the black ball. There are $\binom{y_\textrm{tot}}{y_l}$ (resp., $\binom{g_\textrm{tot}}{g_l}$) ways to choose the set of yellow (resp., green) balls that falls to the left of the black ball. Among the $y_l + g_l$ balls that fall to the left of the black ball, there are $(y_l + g_l)!$ possible permutations. Likewise, there are $(y_\textrm{tot} - y_l + g_\textrm{tot} - g_l)!$ possible permutations among the balls that fall to the right of the black ball. Therefore, the total number of configurations such that $Y_l = y_l$ and $G_l = g_l$ is
$$ \binom{y_\textrm{tot}}{y_l} \binom{g_\textrm{tot}}{g_l} (y_l + g_l)! (y_\textrm{tot} - y_l + g_\textrm{tot} - g_l)!.$$ Given that each arrangement is equally likely, we have that
\begin{multline*}
\P(G_l = g_l, Y_l = y_l) = \frac{ \binom{y_\textrm{tot}}{y_l} \binom{g_\textrm{tot}}{g_l} (y_l + g_l)! (y_\textrm{tot} - y_l + g_\textrm{tot} - g_l)! }{( y_\textrm{tot} + g_\textrm{tot} + 1)! } \\ = \binom{y_\textrm{tot}}{y_l} \binom{g_\textrm{tot}}{g_l} B(g_l + y_l + 1, g_\textrm{tot} - g_l + y_\textrm{tot} - y_l + 1),
\end{multline*}
where $B$ denotes the beta function. Next, for given $\tau \in \{0, \dots, y_\textrm{tot}\}$, we have by the definition of conditional probability that

\begin{equation}\label{combinatorics_1}
\P(G_l = g_l | Y_l \leq \tau) = \frac{\P( G_l = g_l, Y_l \leq \tau)}{\P(Y_l \leq y_l)}.
\end{equation}
We can decompose the numerator of (\ref{combinatorics_1}) as
$$\P( G_l = g_l, Y_l \leq \tau) = \sum_{y_l=0}^\tau \binom{y_\textrm{tot}}{y_l} \binom{g_\textrm{tot}}{g_l} B(g_l + y_l + 1, g_\textrm{tot} - g_l + y_\textrm{tot} - y_l + 1).$$
Meanwhile, because the rank of the black ball among the yellow balls is uniformly distributed marginally,
$$\P(Y_l \leq \tau) = \sum_{y_l = 0}^\tau \P( Y_l = \tau ) =\frac{\tau + 1}{y_\textrm{tot} + 1} .$$ $\square$

% Plugging these expressions into (\ref{combinatorics_1}) yields
%\begin{equation}
%\P(G_l = g_l | Y_l \leq \tau) = \frac{y_\textrm{tot} + 1}{ \tau + 1} \binom{g_\textrm{tot}}{g_l} \sum_{y_l=0}^\tau \binom{y_\textrm{tot}}{y_l} B(g_l + y_l + 1, g_\textrm{tot} - g_l + y_\textrm{tot} - y_l + 1).
% \end{equation} $\square$

Next, in the context of the original problem, the total number of yellow balls $y_\textrm{tot}$ maps to the number of statistics resampled in the first round, $B_1$; the total number of green balls $g_\textrm{tot}$ maps to the number of statistics sampled in the second round, $B_2$; the number of yellow balls to the left of the black ball maps to the rank of the original statistic among the first-round statistics minus 1, $r_1 = \textrm{rank}(t; \mathcal{Y}) - 1$; and the total number of green balls to the left of the black ball maps to the rank of the original statistics among the second-round statistics minus 1, $r_2 = \textrm{rank}(y; \mathcal{Z}) - 1.$ Hence, by the derivation above,

$$\P(r_2 = k , r_1 \leq c ) = \sum_{i=1}^1 $$


\begin{algorithm}
	\SetKwData{rankone}{rank$_1$} \SetKwData{ranktwo}{rank$_2$} \SetKwData{p}{p}
	\caption{Two-tailed checkpoint permutation test}\label{alg:checkpoint_perm_two_sided}
	\Input{$B_1$, $B_2$ and $c$.}
	\Output{Normalized rank \p of ground truth statistic among null statistics.}
	\BlankLine
	Compute ground truth test statistic $t$ and null statistics $y_1, \dots, y_{B_1}$\;
	\rankone $\leftarrow$ $\textrm{rank}(t; \{y_1, \dots, y_{B_1}\})$\;
	\eIf{$\rankone  \leq c $ \textrm{ or } $\rankone \geq B_1 + 1 - c$}{
		Compute fresh null statistics $z_1, \dots, z_{B_2}$\;
		\ranktwo $\leftarrow$ $\textrm{rank}(t; \{z_1, \dots, z_{B_1}\})$\;
		\eIf{$\rankone  \leq c $}{
			\tcp{left-tailed test in round 2}
			$\p \leftarrow \ranktwo/(B_2 + 1)$\;
		}{
			\tcp{$\rankone \geq B_1 + 1 - c$; right-tailed test in round 2}
			$\p \leftarrow (B_2 + 1 - \ranktwo)/(B_2 + 1)$\;
		}
	}{
		\tcp{$c < \rankone < B_1 + 1 - c$; two-tailed test in round 1}
		\p $\leftarrow 2 \min\{\rankone , B_1 + 1 - \rankone\}/(B_1 + 1)$\;
	}
\end{algorithm}



\bibliographystyle{unsrt}
\bibliography{/Users/timbarry/Documents/optionFiles/library.bib}

\end{document}
