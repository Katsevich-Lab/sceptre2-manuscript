\documentclass[12pt]{article}
\usepackage{amsfonts}
\usepackage{amsmath}
\usepackage{graphicx} 
\usepackage{float}
\usepackage[caption = false]{subfig}
\usepackage{/Users/timbarry/Documents/optionFiles/mymacros}

\begin{document}

\noindent
Tim

\begin{center}
\textbf{A strategy for generating negative control pairs: version \# 2}
\end{center}

I describe another, more workable (and simpler) strategy for generating negative control pairs. Again, define the following variables.

\begin{itemize}
	\item The number of negative control gRNAs $N_\textrm{grna}$. Label the NT gRNAs $1, 2, \dots, N_\textrm{grna}$.
	\item The $N_\textrm{cell} \times N_\textrm{gene}$ matrix of gene expressions and the $N_\textrm{cell}$-dimensional gRNA-to-cell assignment vector.
	\item The number of pairs to generate $N_\textrm{pairs}$.
	\item The undercover group size $k \leq N_\textrm{grna}/2$.
	\item The minimum number of treatment cells $N_\textrm{trt}$ and control cells $N_\textrm{cntrl}$ needed for a pair to pass pairwise QC.
\end{itemize}
We proceed in several steps.
\\ \\
\textbf{Step 1: Tabulate the number of of cells with nonzero expression for each individual NT gRNA and gene}. First, we compute an $N_\textrm{grna} \times N_\textrm{gene}$ matrix $M$, where entry $(i,j)$ is the number of cells containing NT gRNA $i$ with nonzero expression of gene $j$. We easily can construct this matrix either in memory or out-of-core by summing over columns of the gene expression matrix.
\\ \\
\textbf{Step 2: Determine if it is feasible to enumerate the possible undercover gRNA groups}. We check the value of $N_\textrm{possible-groups} := \binom{N_\textrm{gene}}{k}$. If $N_\textrm{possible-groups}$ is a huge number (e.g, $ \binom{100}{50} \approx 10^{30}$), then it is not possible to enumerate the possible undercover gRNA groups. If $N_\textrm{possible-groups}$ is small, by contrast (e.g., $\binom{9}{2} = 36$), then it is possible to enumerate the possible undercover gRNA groups. We check if $N_\textrm{possible-groups}$ exceeds some pre-defined threshold (e.g., 20,000), carrying out a different routine in either case. If $N_\textrm{possible-groups} \leq 20,000$, then we proceed to step 3a. Otherwise, we proceed to step 3b. 
\\ \\
\textbf{Step 3a: Enumerate the possible undercover gRNA groups}. If $N_\textrm{possible-groups} \leq 20,000$, we enumerate the possible undercover gRNA groups. We map each possible undercover gRNA group to a length-$k$ vector of integers sorted in increasing order, where the integers represent individual NT gRNAs. For example, the undercover gRNA group containing NT gRNAs 2, 3, and 7 (arbitrarily labeled) would be mapped to the vector $\texttt{[ 2, 3, 7 ]}.$ We then generate the entire set of $N_\textrm{possible-groups}$ length-$k$ vectors containing integers in the range $\{1, \dots, N_\textrm{grna}\}$. We store these vectors in an ordered list $\texttt{x}$. We also set $N_\textrm{grna-groups} = N_\textrm{possible-groups}.$
\\ \\
\textbf{Step 3b: Sample a set of possible undercover gRNA groups}. If $N_\textrm{possible-groups} > 20,000$, then we do not attempt to enumerate the entire set of possible undercover gRNA groups. Instead, we sample a set of undercover gRNA groups. We proceed as follows. First, we estimate the fraction of undercover gRNA-gene pairs (of group size $k$) that passes QC. We do this by pairing a randomly generated undercover gRNA group to a randomly selected gene and checking if that pair passes the pairwise QC threshold. We sample (with replacement) a large number (e.g., $5,000$) undercover gRNA-gene pairs in this way, producing an estimate $\hat{p}$ of the fraction of undercover gRNA-gene pairs that passes QC. We then set the number of gRNA groups to sample $N_\textrm{grna-groups}$ to $$N_\textrm{grna-groups} = \frac{c \cdot N_\textrm{pairs} }{ \hat{p} N_\textrm{genes}},$$ where $c > 1$ is a number that ensures we sample a \textit{conservative} number of gRNA groups (i.e., more than we need). Finally, we sample $N_\textrm{grna-groups}$ gRNA groups by sampling from the set of length $k$ vectors containing integers in the range $\{1, \dots, N_\textrm{grna}\}$ via membership checking sampling.$^*$ We store these $N_\textrm{grna-groups}$ vectors in an ordered list $\texttt{x}$.

$^*$  
\\ \\
\textbf{Step 4: Sample without replacement from the set of undercover gRNA group-gene pairs}. The final step is to sample a set of undercover gRNA group-gene pairs without replacement. Recall that step 3 yields a list $\texttt{x}$ of undercover gRNA groups of length $N_\textrm{grna-groups}$. (This is true whether we have carried out step 3a or step 3b). There are thus $N_\textrm{grna-group} \cdot N_\textrm{gene}$ pairs that we could sample. We map each gRNA group-gene pair in this set of pairs to an integer in the set $\{1, \dots, N_\textrm{gene} \cdot N_\textrm{grna-group} \}.$ The map is defined as follows: for an integer $i \in \{1, \dots, N_\textrm{gene} \cdot N_\textrm{grna-group}\},$ we carry out the integer division $$ \texttt{grna\_group\_idx} = i \%/\% N_\textrm{gene},$$ which defines a gRNA group index. Next, we compute the remainder of this division
$$ \texttt{gene\_idx} = i \%\% N_\textrm{gene}$$ to compute a gene index. Through this map, sampling without replacement from the set of integers $\{1, \dots, N_\textrm{gene} \cdot N_\textrm{grna-group}\}$ is identical to sampling without replacement from the set of undercover gRNA group-gene pairs.

If the number of pairs to sample $N_\textrm{pairs}$ exceeds the number of pairs that we possibly could sample $ N_\textrm{gene} \cdot N_\textrm{grna-group}$, then we iterate through the pairs one-by-one, checking if the pair passes pairwise-QC and, if so, adding it to the set pairs to return. If, on the other hand, the number of pairs to sample $N_\textrm{pairs}$ is less than $N_\textrm{gene} \cdot N_\textrm{grna-group}$, we sample without replacement from the set $\{ 1, \dots, N_\textrm{gene} \cdot N_\textrm{grna-group} \},$ discarding those pairs that do not pass QC. (We can implement this final sampling without replacement step via Fisher-Yates sampling or sparse Fisher-Yates sampling.)


\bibliographystyle{unsrt}
\bibliography{/Users/timbarry/Documents/optionFiles/library.bib}

\end{document}
