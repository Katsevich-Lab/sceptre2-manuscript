\documentclass[12pt]{article}
\usepackage{amsfonts}
\usepackage{amsmath}
\usepackage{graphicx}
\usepackage{algorithm}
\usepackage{algpseudocode}
\usepackage{float}
\usepackage[caption = false]{subfig}
\usepackage{/Users/timbarry/Documents/optionFiles/mymacros}

\begin{document}
\noindent
Tim B.

\begin{center}
\textbf{Inductive sampling without replacement}
\end{center}

\subsection*{Definition and motivation}

I describe ``inductive sampling without replacement,'' a procedure for sharing without replacement (WOR) samples across gene-gRNA pairs containing the same number of negative control (NT) cells but different numbers of treatment cells. Inductive sampling WOR reduces the number of WOR samples that must drawn by a factor equal to the number of gRNAs in the dataset. On a more technical level inductive sampling WOR considerably reduces the number of database queries (specifically, gene expression vector loads) that must be issued when running SCEPTRE out-of-core. Inductive sampling WOR therefore could substantially reduce the compute associated with running SCEPTRE (on low MOI data with marginal resampling), both in memory and out-of-core.

Let $N$ be the number of control cells. Label the control cells by $c_1, c_2, \dots, c_N$. Next, let $M$ be the number of treatment cells containing the the gRNA that infects the greatest number of cells. Label the treatment cells by $t_1, \dots, t_M$. We seek to construct a length-$M$ random sequence $a_1, a_2, \dots, a_M$ that satisfies the following properties:
\begin{itemize}
\item[1.] $a_i \in \{ c_1, \dots, c_N, t_1, \dots, t_i \} $ for all $a_i$ (i.e., the $i$th element of the sequence is a control cell or one of the first $i$ treatment cells).
\item[2.] $a_i \neq a_j$ (i.e., the elements of the sequence are unique).
\item[3.] Letting $A_i$ denote the set containing the first $i$ elements of the sequence (i.e., $A_i := \{ a_1, a_2, \dots, a_i \}$), then
\begin{multline*}
\P(c_1 \in A_i) = \P(c_2 \in A_i) = \dots = \P(c_N \in A_i) = \\ \P(t_1 \in A_i) = \dots = \P(t_i \in A_i) = \frac{i}{i + N}.
\end{multline*}
\end{itemize}

In other words, we seek to construct a sequence of increasing sets $A_1 \subset A_2 \subset \dots \subset A_M$ that satisfies an ``inductive'' WOR sampling property. The first set $A_1$, which contains a single element, contains the control cells and the first treatment cell with equal probability. The second set $A_2$, which contains two elements, contains the control cells and each of the first \textit{two} treatment cells with equal probability, and so on. This property is appealing because it enables us to share random samples across gRNAs. If a given gRNA has $i$ treatment cells, then $A_i$ is a valid WOR sample for this gRNA. Thus, we need only generate a random sequence $a_1 \dots a_M$ once and share this random sequence across all gRNAs.

\subsection*{Constructing an inductive WOR sample}

I describe a strategy for constructing an inductive WOR sample, and I prove its correctness. I describe the procedure inductively.

\textbf{Step 1}. Sample one element from the set $a_1 \leftarrow \{c_1, \dots, c_N, t_1\},$ putting an equal mass of  $1/(N + 1)$ onto each of the elements.

\textbf{Step $i$, for $i \geq 2$}. Let $B_i := \{c_1, \dots, c_N, t_1 \dots, t_{i-1} \} \setminus A_{i-1}$ be the set of ``leftover'' elements that were not selected in step $i-1$. There are $N$ elements in the set $B_i$. Draw an element at random from the set $B_i \bigcup \{ t_i \}$, placing a mass of $\frac{i}{i + N}$ on $t_i$ and a mass of $\left(1 - \frac{i}{i + N} \right)/N$ on each of the elements in $B_i$. Set the sampled element to $a_i$. Continue this process until $i = M$, resulting in a sequence $a_1,\dots,a_M$. This sequence satisfies the desired property stated above, which I now prove.

\textbf{Proof}. I proceed inductively. \underline{Base case}: Let $i = 1$. Then $\P(c_1 \in A_1) = \dots \P(c_N \in A_1) = \P(t_1 \in A_1) = 1/(1+N).$ Next, let $i \in \{2, \dots, M \}$ be given. \underline{Inductive step}: Suppose that 
\begin{multline*}
\P(c_1 \in A_{i}) = \dots = \P(c_N \in A_{i}) = \P( t_1 \in A_{i} ) = \dots \\ = \P(t_{i} \in A_{i}) = \frac{i}{N + i}.
\end{multline*}
We construct $a_{i+1}$ by sampling $t_{i+1}$ with probability $(i+1)/(N + i + 1)$ and the elements of $B_i$ with probability $\left(1 - \frac{i + 1}{N+i + 1}\right)/N$. We have by construction that $$\P(t_{i+1} \in A_{i+1}) = \frac{i+1}{N+i+1}.$$ Next, let $u \in \{c_1, \dots, c_N, t_1, \dots, t_{i} \}$. We compute $\P(u \in A_{i+1})$ using the law of total probability:
\begin{equation}\label{key_eq}
\P(u \in A_{i+1}) = \P(u \in A_{i+1} | u \in A_{i}) \P(u \in A_{i}) + \P(u \in A_{i+1} | u \notin A_{i}) \P(u \notin A_{i}).
\end{equation}
Considering first the lefthand term, we have that $\P(u \in A_{i+1} | u \in A_{i}) = 1,$ as membership in $A_{i}$ implies membership in $A_{i+1}$. Next, $\P(u \in A_{i}) = i/(N + i)$ by the inductive hypothesis. Thus, the left term of (\ref{key_eq}) is
$i/(N + i).$ Next, consider the righthand term. If $u \notin A_{i}$, then $u \in B_k$. Thus, $$ \P(u \in A_{i+1} | u \notin A_{i}) = \left(1 - \frac{i+1}{N+1+i}\right)/N.$$ Furthermore, by the inductive hypothesis, $\P(u \notin A_{i-1}) = 1 - i/(N + i)$. Stringing these pieces together,

\begin{multline*}
\P \left( u \in A_{i+1} \right) = \frac{i}{N + i} + \frac{1}{N} \left(1 - \frac{i+1}{N+1+i}\right) \left( 1 - \frac{i}{N+i} \right) \\ = \frac{i}{N+i} + \frac{1}{N} \left( \frac{N+1+i}{N+1+i} - \frac{i+1}{N+1+i} \right)\left( \frac{N+i}{N+i} - \frac{i}{N+i} \right) \\ = \frac{i}{N+i} + \frac{1}{N} \left( \frac{N}{N+1+i} \right) \left( \frac{N}{N+i} \right) = \frac{i}{N+i} + \left( \frac{1}{N+1+i} \right)\left( \frac{N}{N+i} \right) \\ = \frac{i}{N+i} + \frac{N}{(N+1+i)(N+i)} = \frac{i(N+1+i) + N}{(N+i)(N+1+i)} = \frac{iN + i + i^2 + N}{(N+1)(N+1+i)} \\ = \frac{(N+i)(i+1)}{(N+1)(N+1+i)} = \frac{i+1}{N+1+i}. \\ \square
\end{multline*}


\subsection*{An algorithm for inductive WOR sampling}

I describe an algorithm for inductive WOR sampling using only a $U(0,1)$ random number generator. First, I introduce a distribution --- the ``IWOR distribution.'' The $\text{IWOR}(N, i)$ distribution is a discrete probability distribution that has support $\{ 0, \dots, N \}$ and places mass $i/(i + N)$ on $N$ and mass $\frac{1}{N}\left[1 - i/(i+1)\right]$ on $\{0, 1, \dots, N - 1\}.$ One can sample from the $\text{IWOR}(N,i)$ distribution as follows (Algorithm \ref{alg:iwor_dist}). The algorithm is fast, requiring only an if statement, a multiplication, and a floor operation.

Next, I give an algorithm for constructing an inductive WOR sequence $a_1 a_2 \dots a_M$ given negative control cells $c_1, \dots c_N$ and treatment cells $t_1, \dots, t_M$ (Algothtm \ref{alg:iwor_sample}; the algorithm uses zero-based indexing.) The algorithm is efficient, requiring $O(M)$ time and $O(N + M)$ space. In practice we will index the control cells by $0, 1, \dots, N-1$ and the treatment cells by $N, N+1, \dots, N + M - 1$. Thus the initialization of $r$ step (namely, line 2) can be rewritten as $$ r \leftarrow [0, 1, \dots, N - 1, N],$$ and the final line of the for loop can be rewritten as
$$ r[N] \leftarrow N + i - 1.$$

Suppose we want to construct a sequence $a_1 a_2 \dots a_{m-1} a_m a_{m+1} \dots a_{M-1}a_M$ that satisfies the inductive WOR property from $m$ to $M$. (In our application $m$ and $M$ may be the minimum and maximum number of treatment cells, respectively). We can construct $a_1\dots a_{m-1}$ via standard sampling WOR and then run Algorithm \ref{alg:iwor_sample} to construct $a_m\dots a_M$. A common algorithm for standard WOR sampling is the Fisher-Yates sampler. 

\begin{algorithm}
	\caption{Sampling from the $\textrm{IWOR}(N,i)$ distribution.}\label{alg:iwor_dist}
	\begin{algorithmic}
		\Require $N, i$ 
		\State $u \sim U(0,1)$
		\State $p \leftarrow i/(N+i)$
		\If{$u > 1 - p$}
		\State $d \leftarrow N$
		\Else 
		\State $d \leftarrow \lfloor uN/(1-p) \rfloor $ // floor operator
		\EndIf
		\State \textbf{return} $d$
	\end{algorithmic}
\end{algorithm}

\begin{algorithm}
	\caption{Constructing an inductive WOR sample.}\label{alg:iwor_sample}
	\begin{algorithmic}
		\Require Control cells $c_1, \dots, c_N$ and treatment cells $t_1, \dots, t_M$.
		\State Initialize $v \leftarrow \textrm{vector}(M).$
		\State Initialize $r \leftarrow [ c_1, c_2, \dots, c_N, t_1 ].$
		\For{i = 1 $\dots$ M }
		\State $\texttt{pos} \sim \textrm{IWOR}(N, i)$ // sample a position within $v$
		\State $v[i-1] \leftarrow r[\texttt{pos}] $ // extract the element at that position
		\State $ r[\texttt{pos}] \leftarrow r[N]$ // move the rightmost entry of $r$ to position \texttt{pos}
		\State $r[N] \leftarrow t_{i+1}$ // update the rightmost entry of $r$ with $t_{i+1}$
		\EndFor
		\State \textbf{return} $v$
	\end{algorithmic}
\end{algorithm}


\begin{algorithm}
\caption{Hybrid Fisher-Yates/IWOR sampler}
\begin{algorithmic}
\Require Control cells $c_1, \dots, c_N$ and treatment cells $t_1, \dots, t_M$; the minimum number of treatment cells $m$.
\State $x \leftarrow [c_1, \dots, c_N, t_1, \dots, t_m].$
\State $v \leftarrow \textrm{vector}(M)$ 
\For{$i = 1, \dots, m$} // Fisher-Yates step:
\State $u \sim \textrm{Unif}(\{0, 1, \dots, N + m - i\})$
\State $\textrm{Swap}(x[N + m - i], x[u])$
\EndFor
\State // entries $x[0, \dots, N - 1]$ are $N$ leftovers from $x$
\State // entries $x[N, \dots, N + m - 1]$ are $m$ samples WOR from $x$
\\
\For{$i = 0, \dots, m - 1$}
\State $v[i] \leftarrow x[i + N]$
\EndFor

\\ 
\State $x[N] \leftarrow t_{m+1}$ // initialize inductive WOR step
\For{i = $m+1, \dots, M$}
\State \texttt{pos} $\sim$ \text{IWOR}(N, i)
\State $v[i - 1] \leftarrow x[\texttt{pos}]$
\State $x[\texttt{pos}] \leftarrow x[N]$
\State $x[N] \leftarrow t_{i+1}$
\EndFor
\State \textbf{return} $v$
\end{algorithmic}
\end{algorithm}

\subsection*{A variant on inductive sampling without replacement}

I consider a variant on inductive sampling WOR, relevant to carrying out the undercover analysis in low MOI and the undercover/discovery analysis in high MOI. In these settings the \textit{total number} of cells stays the same, while the number of treatment cells and control cells changes. Let $c_1, \dots, c_{N-M}$ be the control cells and $t_1, \dots, t_M$ be the treatment cells. We seek to construct a sequence $a_1 a_2 \dots a_N$ such that $A_i = \{a_1, \dots, a_i\}$ is a valid WOR sample for $t_1, \dots, t_i$ and $c_1, \dots, c_{N-i}$. 



\bibliographystyle{unsrt}
\bibliography{/Users/timbarry/Documents/optionFiles/library.bib}

\end{document}
